\documentclass[11pt]{beamer}
\usetheme{Warsaw}
\usepackage[utf8]{inputenc}
\usepackage[german]{babel}
\usepackage[T1]{fontenc}
\usepackage{amsmath}
\usepackage{amsfonts}
\usepackage{amssymb}
\usepackage{graphicx}
\usepackage{caption}
\usepackage{subcaption}
\usepackage{float}

\author{Stefan Zaufl, Christian Brändle, Dominik Schörkhuber}
\title{3D Vision UE}
%\setbeamercovered{transparent} 
%\setbeamertemplate{navigation symbols}{} 
%\logo{} 
%\institute{} 
%\date{} 
%\subject{} 

\begin{document}

\begin{frame}
\titlepage
\end{frame}

%\begin{frame}
%\tableofcontents
%\end{frame}

\begin{frame}{Mann-Modell}
	\begin{columns}
	\column{.3\textwidth}
		\includegraphics[width=3.5cm]{images/Mann_Original.jpg}
	\column{.7\textwidth}
		\begin{block}{}
			\begin{itemize}
			\item Kleine Figur aus Holz
			\item Hat durch die Maserung eine gute Textur
			\item Form annähernd eine Kugel
			\item Problemstellen: konkave Regionen (z.B. bei den Armen) 
			\end{itemize}
		\end{block}
	\end{columns}
\end{frame}

\begin{frame}{Mann-Modell: Scan}
	\begin{figure}
		\begin{subfigure}{0.4\textwidth}
			\includegraphics[width=\textwidth]{images/Mann_Front}
		\end{subfigure}
		\begin{subfigure}{0.4\textwidth}
			\includegraphics[width=\textwidth]{images/Mann_Back}
		\end{subfigure}
	\end{figure}
\end{frame}

\begin{frame}{Mann-Modell: 123D Catch}
	\begin{figure}
		\begin{subfigure}{0.4\textwidth}
			\includegraphics[width=\textwidth]{images/Mann_SFM_Front}
		\end{subfigure}
		\begin{subfigure}{0.4\textwidth}
			\includegraphics[width=\textwidth]{images/Mann_SFM_Back}
		\end{subfigure}
	\end{figure}
\end{frame}

\begin{frame}{Mann-Modell: Evaluierung}

	\begin{block}{Messpunkte}
		\begin{itemize}
		\item Fußlänge
		\item Armlänge
		\item Hosenbund
		\end{itemize}
	\end{block}
	
	\begin{block}{Ergebnisse}
		\begin{itemize}
		\item $\varnothing$ Fehler Scan: $\pm 0,93$mm
		\item $\varnothing$ Fehler 123D Catch: $\pm 2,62$mm
		\item Volumen Scan: $544.250,14mm^3$
		\item Volumen 123D Catch: $583.912,27mm^3$
			\begin{itemize}
			\item Differenz: $39.662,13mm^3(7,28\%)$
			\end{itemize}
		\end{itemize}
	\end{block}
	
\end{frame}

\begin{frame}{Budha}
\center
	\includegraphics[width=3cm]{images/budha/Budha_original.jpg}
	\begin{block}{Objektbeschreibung}
		\begin{itemize}
			\item rauhe Oberfläche
			\item gesprenkelte Texturierung
			\item viele feine Details
			\item matt \& dunkel
		\end{itemize}
	\end{block}

\end{frame}

\begin{frame}{Finales Modell - Budha}
\center
	\includegraphics[height=4.3cm]{images/budha/Budha_3DScan_Untextured_2.png}
	\includegraphics[height=4.3cm]{images/budha/Budha_3DScan_Untextured.png}
	\begin{block}{Objektbeschreibung}
		\begin{itemize}
			\item Global, Manuell Registriert, Vereinigt
			\item Meshdoctor zur Analyse
			\item manuelle Entfernung von Meshfehlern mit anschließendem Schließen
			\item Smoothing
		\end{itemize}
	\end{block}
\end{frame}

\begin{frame}{123d Catch - Budha}
\center
	\includegraphics[height=4.5cm]{images/budha/Budha_SfM_Textured.png}
	\includegraphics[height=4.5cm]{images/budha/Budha_SfM_Textured_2.png}
	\begin{block}{Objektbeschreibung}
		\begin{itemize}
			\item Viel Textur und Geometrie -> gute Features
			\item Nachbearbeitung: Trennen der Figur vom Untergrund \& Loch schließen
		\end{itemize}
	\end{block}
\end{frame}

\begin{frame}{Evaluierung - Budha}

	\includegraphics[height=2.5cm]{images/budha/Budha_original.jpg}
	\includegraphics[height=2.5cm]{images/budha/Budha_3DScan_Untextured_2.png}
	\includegraphics[height=2.5cm]{images/budha/Budha_3DScan_Untextured.png}
	\includegraphics[height=2.5cm]{images/budha/Budha_SfM_Textured.png}
	\includegraphics[height=2.5cm]{images/budha/Budha_SfM_Textured_2.png}
	
	\begin{block}{Messpunkte}
		\begin{itemize}
			\item Fußpunkt
			\item Haaransatz
		\end{itemize}
	\end{block}
	\begin{block}{Messfehler}
		Abweichung von $5 \%$ im Volumen
	\end{block}
\end{frame}

\begin{frame}{Sparschwein}
\center
	\includegraphics[height=5cm]{images/sparschwein/photo.jpg}
	\begin{block}{Objektbeschreibung}
		\begin{itemize}
			\item Einfache Geometrie
			\item Sehr dunkel -> Laser Intensität
			\item Reflektierende Teile
		\end{itemize}
	\end{block}

\end{frame}

\begin{frame}{Finales Modell - Sparschwein}
\center
	\includegraphics[height=4.8cm]{images/sparschwein/final2}
	\includegraphics[height=4.8cm]{images/sparschwein/final3}
	\begin{block}{Bearbeitungsschritte}
		\begin{itemize}
			\item Global, Manuell Registriert, Vereinigt
			\item Meshdoctor
			\item Fehlende Geometrie an den Zehen wiederhergestellt
			\item Sanding
		\end{itemize}
	\end{block}
\end{frame}

\begin{frame}{123d Catch - Sparschwein}
\center
	\includegraphics[height=5cm]{images/sparschwein/123d_Vergleich}
	\begin{block}{Objektbeschreibung}
		\begin{itemize}
			\item Wenig Textur -> wenig Features :(
			\item Erst durch manuelle Registrierung brauchbar
		\end{itemize}
	\end{block}
\end{frame}

\begin{frame}{Evaluierung - Sparschwein}
	\begin{block}{Messpunkte}
		\begin{itemize}
			\item Nasenbreite
			\item Ohrenbreiten
			\item Münzschlitzlänge
			\item Münzauswurf Durchmesser
			\item Schriftgröße
		\end{itemize}
	\end{block}
	\begin{block}{Messfehler}
		Abweichung von $\pm1$mm
	\end{block}
\end{frame}

\begin{frame}{Nachbearbeitung in GeoMagic}

\begin{block}{Nachbearbeitungsschritte}
\begin{enumerate}
\item Registrierung
\begin{enumerate}
\item Globale Registrierung in Gruppen
\item Manuelle Registrierung der Gruppen, n-Punkt Registrierung
\end{enumerate}
\item Meshgenerierung
\begin{enumerate}
\item Vereinigen der registrierten Meshes
\end{enumerate}
\item Nachbearbeitung
\begin{enumerate}
\item Nachbearbeitung: 2D-Mannigkeitsfehler entfernen, Glätten, Spitzen entfernen, Mesh-Doctor, Löcher füllen
\item Feine Nachbearbeitung: Sandpapier
\end{enumerate}
\end{enumerate}
\end{block}

\end{frame}

\begin{frame}{Registrierung}

\begin{block}{globale Registrierung}
\begin{figure}[H]
\centering
\includegraphics[width=0.4\textwidth]{images/GeoMagicBudhaPictures/Budha_Scans_Aufrecht_globalRegistration_2.PNG}
\caption{Globale Registrierung - Falschfarbendarstellung verschiedener Range Maps}
\label{fig:budhaGlobal}
\end{figure}
\end{block}

\end{frame}

\begin{frame}{Registrierung}

\begin{block}{manuelle n-Punkt Registrierung}
\begin{figure}[H]
\centering
\includegraphics[width=0.65\textwidth]{images/GeoMagicBudhaPictures/Budha_Scans_Aufrecht_manualRegistration_03.PNG}
\caption{Manuelle Registrierung mittels verteilter Punkte}
\label{fig:budhaManual}
\end{figure}
\end{block}

\end{frame}


\begin{frame}{Meshverarbeitung}

\begin{block}{Mesh Doctor}
\begin{figure}[H]
\centering
\includegraphics[width=0.5\textwidth]{images/GeoMagicBudhaPictures/Budha_MeshDoctor.PNG}
\caption{Mesh-Doktor - Highlighting von defekten oder zu bearbeitenden Stellen}
\label{fig:budhaMeshDoc}
\end{figure}
\end{block}

\end{frame}

\begin{frame}{Meshverarbeitung}

\begin{block}{Löcher schließen}
\begin{table}[h]
	\begin{center}
		\begin{tabular}{| c | c |}
			\hline
			offen & geschlossen \\
			\hline
			\hline
			& \\
			\includegraphics[width=0.35\textwidth]{./Images/GeomagicBudhaPictures/Budha_SfM_BottomHole_2.PNG} & \includegraphics[width=0.345\textwidth]{./Images/GeomagicBudhaPictures/Budha_SfM_BottomHoleClosed_2.PNG} \\
			\hline					  
		\end{tabular}
	\end{center}
	\caption{Manuelles Schließen der Grundfläche einer Figur, links vorher, rechts nachher}
	\label{tab:BudhaHole}
\end{table}
\end{block}

\end{frame}

\begin{frame}{Meshverarbeitung}

\begin{block}{Glätten}
\begin{table}[h]
	\begin{center}
		\begin{tabular}{| c | c |}
			\hline
			vor Glättung & nach Glättung \\
			\hline
			\hline
			& \\
			\includegraphics[width=0.30\textwidth]{./Images/GeomagicBudhaPictures/Budha_Presmooth.PNG} & \includegraphics[width=0.285\textwidth]{./Images/GeomagicBudhaPictures/Budha_Postsmooth.PNG} \\
			\hline					  
		\end{tabular}
	\end{center}
	\caption{Manuelles Glätten einer Figur, links vorher, rechts nachher}
	\label{tab:BudhaSmooth}
\end{table}
\end{block}

\end{frame}

\end{document}