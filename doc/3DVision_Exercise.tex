\documentclass[]{article}

\usepackage{ngerman}
\usepackage{comment}
\usepackage[ngerman]{babel}
\usepackage[utf8]{inputenc}
\usepackage[T1]{fontenc}
\usepackage{amsmath}
\usepackage{graphicx}


% Title Page
\title{3D Vision - }
\author{Christian Br\"andle, Matr.Nr.: 1428543}

\begin{document}

\maketitle

%\begin{abstract}
%\end{abstract}

\section{3D Scanning}

Die Scandaten wurden $25$ Körperscans und $15$ Scans des Kopfes vorgenommen.
Die Scans des Kopfes wurden mit einer anderen Linse vorgenommen um eine erhöhte Auflösung im Speziellen des Gesichtes zu erreichen.

\subsection{Noise removal, registration and merging}

Zunächst wurden alle $40$ Scans gruppiert und manuell in den Gruppen zueinander registriert, da eine automatische Registrierung in den meisten Fällen fehlschlug.

Danach wurden die einzelnen Gruppen zueinander registriert um eine Gesamtdatenbasis des Modells zu erhalten.

Im Anschluss wurden auf den einzelnen Scans Fehler an den Scanrändern und bzw. überflüssige Teile der Scans von der Scanplatte entfernt.

\subsection{Waterproof 3D Model}

\subsection{Evaluation}


\section{Structure from Motion}

\subsection{Take photographs}

\subsection{Noise removal}


\section{Comparison}




\begin{thebibliography}{1}

  \bibitem{STF} Ana Paula B. Lopes, Rodrigo S. Oliveira, Jussara M. de Almeida, Arnaldo de A. Araujo {\em Spatio-Temporal Frames in a Bag-of-visual-features Approach for Human Actions Recognition} Computer Science Department, Federal University of Minas Gerais – UFMG – Belo Horizonte, MG, Brazil
\end{thebibliography}        

		
\end{document}
